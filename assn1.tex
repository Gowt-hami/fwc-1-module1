% #######################################
% ########### FILL THESE IN #############
% #######################################
\def\mytitle{MATRICES}
\def\mykeywords{}
\def\myauthor{GOWTHAMI MANDAVA}
\def\contact{gowthamimandava999@gmail.com}
\def\mymodule{ Future Wireless Communication(FWC22012)}
% #######################################
% #### YOU DON'T NEED TO TOUCH BELOW ####
% #######################################
\documentclass[10pt, a4paper]{article}
\usepackage[a4paper,outer=1.5cm,inner=1.5cm,top=1.75cm,bottom=1.5cm]{geometry}
\twocolumn
\usepackage{circuitikz}
\usepackage{graphicx}
\graphicspath{{./images/}}
%colour our links, remove weird boxes
\usepackage[colorlinks,linkcolor={black},citecolor={blue!80!black},urlcolor={blue!80!black}]{hyperref}
%Stop indentation on new paragraphs
\usepackage[parfill]{parskip}
%% Arial-like font
\usepackage{lmodern}
\renewcommand*\familydefault{\sfdefault}
%Napier logo top right
\usepackage{watermark}
%Lorem Ipusm dolor please don't leave any in you final report ;)
\usepackage{karnaugh-map} 
\usepackage{tabularx}
\usepackage{lipsum}
\usepackage{xcolor}
\usepackage{listings}
%give us the Capital H that we all know and love
\usepackage{float}
%tone down the line spacing after section titles
\usepackage{titlesec}
%Cool maths printing
\usepackage{amsmath}
%PseudoCode
\usepackage{algorithm2e}

\titlespacing{\subsection}{0pt}{\parskip}{-3pt}
\titlespacing{\subsubsection}{0pt}{\parskip}{-\parskip}
\titlespacing{\paragraph}{0pt}{\parskip}{\parskip}
\newcommand{\figuremacro}[5]{
    \begin{figure}[#1]
        \centering
        \includegraphics[width=#5\columnwidth]{#2}
        \caption[#3]{\textbf{#3}#4}
        \label{fig:#2}
    \end{figure}
}


 \lstset{
frame=single, 
breaklines=true,
columns=fullflexible
}

\thiswatermark{\centering \put(1,-110){\includegraphics[scale=0.05]{IIT_logo.png}} }
\title{\mytitle}
\author{\myauthor\hspace{1em}\\\contact\\IITH\hspace{0.5em}-\hspace{0.5em}\mymodule}
\date{}
\hypersetup{pdfauthor=\myauthor,pdftitle=\mytitle,pdfkeywords=\mykeywords}
\sloppy
% #######################################
% ########### START FROM HERE ###########
% #######################################
\begin{document}
 \maketitle
 \tableofcontents
 
    
 

 
    
    
    
 
 \Large\section{Problem}
 Q.Find the equation of the line  parallel to the line 3x-4y+2=0 and passing through the point (-2,3).
 
    
   


 \section{Solution}
 \begin{center}
 Given equation is 3x-4y+2=0
 \\the line parallel to given equation is 3x-4y+k=0
\\the parallel line passing through point(-2,3) 
        \\ 3(-2)-4(3)+k=0
         \\-6-12+k=0
          \\ k=18
       \\ therefore,the equation parallel to the given equation and passing through the point(-2,3) is 3x-4y+18=0
 \end{center}
 \section{Plot}
         \centering
        \includegraphics[scale=0.25]{Figure_1.png}

 
     
   



  \section{Software}
  We can get the parallel equation of given equation and the plot of two equtions by executing the following code:
  
\begin{lstlisting}

https://github.com/Gowt-hami/fwc-1-module1/blob/main/par.py
\end{lstlisting}
 





    

    
 


\end{document}