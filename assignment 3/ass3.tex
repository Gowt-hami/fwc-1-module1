% #######################################
% ########### FILL THESE IN #############
% #######################################
\def\mytitle{KARNAUGH MAP(AVR gcc)}
\def\mykeywords{}
\def\myauthor{GOWTHAMI MANDAVA}
\def\contact{gowthamimandava999@gmail.com}
\def\mymodule{ Future Wireless Communication(FWC22012)}
% #######################################
% #### YOU DON'T NEED TO TOUCH BELOW ####
% #######################################
\documentclass[10pt, a4paper]{article}
\usepackage[a4paper,outer=1.5cm,inner=1.5cm,top=1.75cm,bottom=1.5cm]{geometry}
\twocolumn
\usepackage{circuitikz}
\usepackage{graphicx}
\graphicspath{{./images/}}
%colour our links, remove weird boxes
\usepackage[colorlinks,linkcolor={black},citecolor={blue!80!black},urlcolor={blue!80!black}]{hyperref}
%Stop indentation on new paragraphs
\usepackage[parfill]{parskip}
%% Arial-like font
\usepackage{lmodern}
\renewcommand*\familydefault{\sfdefault}
%Napier logo top right
\usepackage{watermark}
%Lorem Ipusm dolor please don't leave any in you final report ;)
\usepackage{karnaugh-map} 
\usepackage{tabularx}
\usepackage{lipsum}
\usepackage{xcolor}
\usepackage{listings}
%give us the Capital H that we all know and love
\usepackage{float}
%tone down the line spacing after section titles
\usepackage{titlesec}
%Cool maths printing
\usepackage{amsmath}
%PseudoCode
\usepackage{algorithm2e}

\titlespacing{\subsection}{0pt}{\parskip}{-3pt}
\titlespacing{\subsubsection}{0pt}{\parskip}{-\parskip}
\titlespacing{\paragraph}{0pt}{\parskip}{\parskip}
\newcommand{\figuremacro}[5]{
    \begin{figure}[#1]
        \centering
        \includegraphics[width=#5\columnwidth]{#2}
        \caption[#3]{\textbf{#3}#4}
        \label{fig:#2}
    \end{figure}
}


 \lstset{
frame=single, 
breaklines=true,
columns=fullflexible
}

\thiswatermark{\centering \put(1,-110){\includegraphics[scale=0.05]{IIT_logo.png}} }
\title{\mytitle}
\author{\myauthor\hspace{1em}\\\contact\\IITH\hspace{0.5em}-\hspace{0.5em}\mymodule}
\date{}
\hypersetup{pdfauthor=\myauthor,pdftitle=\mytitle,pdfkeywords=\mykeywords}
\sloppy
% #######################################
% ########### START FROM HERE ###########
% #######################################
\begin{document}
 \maketitle
 \begin{abstract}
     %Replace the lipsum command with actual text 
  This document shows the truth table and logic diagram of given boolean function by using KMap. 
 \end{abstract}
    
 

 
    
    
    
 
 \section{Components}
 
     \begin{tabularx}{0.4\textwidth} {  
  | >{\centering\arraybackslash}X  
  | >{\centering\arraybackslash}X  
  | >{\centering\arraybackslash}X |}
  \hline
\textbf{Component} &  \textbf{Value} & \textbf{Quantity}\\
\hline
Arduino & UNO & 1 \\  
\hline
Resistor& 220ohm & 1 \\ 
\hline
Bread board & - & 1 \\
\hline
Jumber wires & M-M & 20\\
\hline
Led & - & 1\\
\hline
\end{tabularx}


    




 \section{Logic}
 The circuit takes 4-bit number from (0-7) as input W,X,Y,Z and produces the F as output according to the logic given in table 1.
\begin{table}[htbp]
 \begin{center}
    \begin{tabular}{|l|c|c|c|c|c|c|c|c|} \hline 
  \textbf{W}& \textbf{X}& \textbf{Y} & \textbf{Z} &\textbf{F(W,X,Y,Z)} \\
 \hline
0 & 0 & 0 & 0 & 0\\  
\hline
0 & 0 & 0 & 1 & 1 \\ 
\hline
0 & 0 & 1 & 0 & 0 \\
\hline
0 & 0 & 1 & 1 & 0 \\
\hline
0 & 1 & 0 & 0 & 0 \\  
\hline
0 & 1 & 0 & 1 & 1 \\ 
\hline
0 & 1 & 1 & 0 & 1 \\
\hline
0 & 1 & 1 & 1 & 1 \\
\hline
1 & 0 & 0 & 0 & 0 \\
\hline
1 & 0 & 0 & 1 & 1 \\
\hline
1 & 0 & 1 & 0 & 1 \\
\hline
1 & 0 & 1 & 1 & 1 \\
\hline
1 & 1 & 0 & 0 & 0 \\
\hline
1 & 1 & 0 & 1 & 1 \\
\hline
1 & 1 & 1 & 0 & 1 \\
\hline
1 & 1 & 1 & 1 & 1 \\
\hline 
\end{tabular}   
\end{center}
\caption{\label{table:dummytable} }
\end{table}

 
   
  

    
\section{Kmap}

Using the boolean logic output F can be expressed in terms of the inputs W,X,Y,Z with the help of the following Kmap.
\\
\\
\begin{karnaugh-map}[4][4][1][$YZ$][$WX$]
        \minterms{1,1,5,6,7,9,10,11,13,14,15}
        \maxterms{0,2,3,4,8,12}
        \implicant{1}{9}
        \implicant{7}{14}
        \implicant{15}{10} 
    \end{karnaugh-map}
\\
The boolean expression for the output F is obtained in the form of POS after minimizing the Kmap maxterm implicants.
\\
\begin{center}
    F=XY'Z+X'Y'Z+W'XY+WX'Y+WXY
\end{center}

 
     
    \section{Hardware Connection}


    
    \begin{table}[htbp]
 \begin{center}
    \begin{tabular}{|l|c|c|c|c|c|c|} \hline 
  \textbf{Arduino}& \textbf{6} & \textbf{7} & \textbf{8}&\textbf{9}&\textbf{5} &\textbf{GND} \\
   \hline
 breadboard & 0/1 & 0/1 & 0/1  & - & -\\ 
 \hline 
led & - & - & - & +ve & -ve \\ 
 \hline 
\end{tabular}   
\end{center}
\caption{\label{table:dummytable} }
\end{table}
Give the connections as per Table 2. For taking the inputs connect 5V of arduino to +ve line of bread board to consider it as logic 'HIGH'.connect GND pin of arduino to -ve line of bread board to consider it as logic 'LOW'.
\\
\\
For example if the inputs W,X,Y,Z are connected 1,0,1,1respectively the output should be 1 i.e., the LED connected to the 5th pin should glow.
\\
\\
In the another case if we connect the inputs W,X,Y,Z to 1,1,0,0 respectively the output should be 0 i.e., the LED connected to 5th pin should turn off

The circuit implementation of the above function is given in figure 1.




  \section{Software}
  1.Connect the arduino to the USB port of computer
  \\
  \\2.Download the follwing code
  \\
  \begin{lstlisting}
 
  \end{lstlisting}
  https://github.com/Gowt-hami/fwc-1-module1/blob/main/assignment%203/main.c
  3.Upload the code into the arduino board.
  \\
  \\4.The output '1' is represented as the state:'LED ON' and '0' is represented as the state 'LED OFF' 
 
      \begin{circuitikz} \draw
       (0,6) node[and port,number inputs=2]  (and1) {}
(0,4) node[and port,number inputs=2]  (and2) {}
(0,2) node[and port,number inputs=2]  (and3) {}
(2,4) node[or port, number inputs=3] (or) {}
(and1.out) -- (or.in 1)
(and2.out) -- (or.in 2)
(and3.out) -- (or.in 3);
\node[left] at (and1.in 1) {\(Z\)};
\node[left] at (and1.in 2) {\(Y\)};
\node[left] at (and1.in 2)[ocirc] {};
\node[left] at (and2.in 1) {\(Y\)};
\node[left] at (and2.in 2) {\(W\)};
\node[left] at (and2.in 2)[] {};
\node[left] at (and3.in 1) {\(Y\)};
\node[left] at (and3.in 2) {\(X\)};
\node[right] at (or.out) {\((F=Y'Z+YX+YW)\)};
\end{circuitikz}
   





    

    
 


\end{document}